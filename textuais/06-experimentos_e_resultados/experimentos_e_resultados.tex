%%
%% Capítulo 6: Experimentos e Resultados
%%

\mychapter{Experimentos e Resultados}
\label{Cap:ExperimentosResultados}

Após a implementação completa do sistema, realizei uma série de testes em bancada para validar o funcionamento da solução proposta. Este capítulo apresenta os experimentos realizados em minha casa, a metodologia empregada e os resultados obtidos, demonstrando a viabilidade técnica do sistema desenvolvido.

\section{Ambiente de Testes em Bancada}

Todos os experimentos foram realizados em minha bancada de trabalho em casa, onde montei um setup de testes com todos os componentes do sistema. A controladora DigiProx SA-202 foi conectada a um LED vermelho na saída NO (Normally Open - normalmente aberta)\footnote{Contato elétrico que permanece aberto em estado de repouso e fecha quando ativado, oposto ao NC (Normally Closed).} para simular o acionamento de uma fechadura elétrica. Quando uma tag autorizada era lida, o LED acendia por 3 segundos, simulando o tempo que uma porta ficaria destravada.

O sistema ficou montado na bancada durante várias semanas, permitindo testes contínuos sempre que eu tinha tempo livre. Durante esse período, realizei centenas de leituras com diferentes tags RFID que tinha disponível, incluindo cartões de acesso antigos, chaveiros e até alguns adesivos RFID que comprei para os testes.

\section{Metodologia de Testes em Casa}

Organizei meus testes de forma metódica, mesmo sendo realizados em casa. Primeiro, testei cada componente individualmente para garantir que estavam funcionando. Depois, fui integrando aos poucos: Arduino com RDM6300, depois adicionei o ESP8266, e finalmente a conexão com o Firebase.

Para medir o desempenho, usei ferramentas simples mas eficazes. O monitor serial do Arduino IDE\footnote{Integrated Development Environment - ambiente de desenvolvimento integrado oficial do Arduino, que inclui editor de código e monitor serial.} me permitia ver os tempos de resposta em milissegundos. Para verificar a latência do Firebase, usei timestamps\footnote{Marca temporal que registra o momento exato em que um evento ocorreu, geralmente em formato Unix ou ISO 8601.} no próprio console do Firebase e comparava com o horário local. Não era uma medição científica perfeita, mas foi suficiente para validar que o sistema funcionava dentro dos parâmetros esperados.

\section{Experimento 1: Validação da Leitura RFID}

O primeiro experimento realizado teve como objetivo validar a capacidade do sistema de ler corretamente as tags RFID de 125 kHz. Para isso, utilizei um conjunto de 25 cartões diferentes, todos compatíveis com o padrão EM4100\footnote{Protocolo de comunicação para tags RFID de 125 kHz, desenvolvido pela EM Microelectronic, amplamente utilizado em sistemas de controle de acesso.}. Uma vantagem importante foi que cada cartão tinha seu código hexadecimal gravado em sua superfície, permitindo conferir se o sistema estava lendo corretamente os valores.

Durante os testes, cada cartão foi apresentado ao leitor em diferentes distâncias e ângulos. O sistema conseguiu ler com sucesso todos os 25 cartões, resultando em uma taxa de sucesso de 100\%. Além disso, verifiquei que os códigos lidos pelo sistema correspondiam exatamente aos valores impressos nos cartões, confirmando a precisão da leitura.

A distância ideal de leitura ficou entre 2 e 4 centímetros, com algumas leituras bem-sucedidas ocorrendo até 5 centímetros de distância. Essa variação na distância é normal para sistemas RFID de 125 kHz e está dentro dos parâmetros esperados para esta tecnologia.

\begin{figure}[htbp!]
\centering
\includegraphics[width=0.8\textwidth]{pre-textuais/figuras/cartoesRFID.JPG}
\caption{Cartões RFID utilizados nos testes com códigos hexadecimais impressos}
\label{fig:cartoes_rfid}
\end{figure}
\section{Experimento 2: Teste de Integração com Firebase}

O segundo experimento focou na validação da comunicação entre o ESP8266 e o Firebase Realtime Database\footnote{Banco de dados NoSQL hospedado na nuvem que permite sincronização de dados em tempo real entre clientes.}. Para este teste, configurei o sistema para enviar automaticamente os dados de cada leitura RFID para o banco de dados, monitorando em tempo real o status das transmissões através do console do Firebase.

Durante um período de 4 horas de teste contínuo, realizei aproximadamente 200 leituras com diferentes cartões. Dessas tentativas, 198 foram transmitidas com sucesso na primeira tentativa, representando uma taxa de sucesso de 99\%. As 2 falhas que ocorreram foram devido a breves interrupções na conexão Wi-Fi, mas o sistema de retry automático implementado conseguiu reenviar os dados perdidos assim que a conexão foi restabelecida.

Todos os dados enviados apareceram corretamente no console do Firebase, com os timestamps correspondendo ao momento exato da leitura. Isso confirmou que o sistema estava funcionando conforme esperado, registrando cada acesso de forma confiável na nuvem.

\section{Experimento 3: Operação Paralela com Controladora Original}

O teste mais importante foi verificar se meu sistema interferia de alguma forma com a controladora original. Conectei ambos os leitores em paralelo na mesma antena e comecei a fazer leituras sucessivas.

Durante vários dias de teste, fiz centenas de leituras com as tags que tinha disponível. Em todos os casos, quando aproximava uma tag cadastrada na controladora, o LED conectado na saída NO acendia (simulando o destravamento da porta) e, simultaneamente, o sistema enviava os dados para o Firebase. Quando usava uma tag não cadastrada, o LED permanecia apagado mas o Firebase ainda registrava a tentativa de acesso, o que é ótimo para auditoria.

Fiz também o teste crucial: desconectei completamente meu sistema (Arduino e ESP8266) e a controladora continuou funcionando perfeitamente, com o LED respondendo normalmente às tags autorizadas. Isso comprovou que minha solução é realmente não invasiva e não cria nenhuma dependência.

\section{Experimento 4: Teste de Resistência}

Para testar a resistência do sistema, deixei tudo ligado continuamente por vários dias. Não tinha como automatizar completamente o teste, então sempre que passava pela bancada, fazia algumas leituras com diferentes cartões.

Durante uma semana de testes intermitentes, realizei centenas de leituras no total. O sistema não travou nenhuma vez e não percebi degradação no desempenho. Os componentes mantiveram temperatura normal durante toda a operação - tanto o Arduino quanto o ESP8266 permaneceram em temperatura ambiente, sem nenhum aquecimento perceptível.

Um detalhe interessante que observei: o sistema continuou funcionando mesmo quando meu roteador Wi-Fi reiniciou durante uma queda de energia. O ESP8266 se reconectou automaticamente quando a rede voltou e enviou os dados que estavam no buffer, exatamente como eu havia programado.

\section{Experimento 5: Testando a Interface Web}

A interface web que criei no ESP8266 foi muito útil durante os testes. Conseguia acessar digitando o IP do ESP8266 no navegador de qualquer dispositivo conectado na mesma rede Wi-Fi.

Testei nos dispositivos que tinha disponível: meu notebook com Chrome e o celular com Safari. Ambos funcionaram perfeitamente, exibindo a interface sem problemas de compatibilidade. O endpoint /setTag foi particularmente útil - podia simular o envio de uma tag para o Firebase sem precisar pegar um cartão físico, bastava digitar algo como "http://192.168.0.105/setTag?code=ABCD1234" no navegador.

A resposta era praticamente instantânea, com a página carregando imediatamente após a requisição.


\section{Resultados Consolidados}

Os experimentos realizados demonstraram que o sistema desenvolvido atende plenamente aos requisitos estabelecidos no projeto. A tabela abaixo apresenta um resumo dos principais resultados obtidos:

\begin{table}[htbp]
\centering
\caption{Resumo dos resultados experimentais}
\label{tab:resultados}
\begin{tabular}{|l|c|}
\hline
\textbf{Métrica} & \textbf{Valor Obtido} \\
\hline
Taxa de leitura RFID & 100\% (25/25 cartões) \\
\hline
Taxa de transmissão Firebase & 99\% (198/200 tentativas) \\
\hline
Distância de leitura & 2-5 cm \\
\hline
Tempo de operação contínua & 7 dias sem falhas \\
\hline
Compatibilidade navegadores & Chrome, Vivaldi, Firefox e Brave \\
\hline
\end{tabular}
\end{table}

\section{Resultados da Implementação}
\label{sec:resultados-implementacao}

\subsection{Funcionalidades Alcançadas}

Após toda a implementação, consegui criar um sistema que superou minhas expectativas iniciais. A leitura de tags RFID ficou extremamente confiável, funcionando perfeitamente com todas as tags de 125 kHz que testei, com alcance efetivo de até 5 centímetros. O processamento é praticamente instantâneo - o registro aparece no Firebase quase imediatamente após a leitura.

Todos os acessos são registrados permanentemente no Firebase, criando um histórico completo que a controladora original não oferece. A interface web que desenvolvi permite monitoramento remoto de qualquer lugar, e o mais importante: tudo isso funciona em paralelo com o sistema original, sem nenhuma interferência.

\subsection{Vantagens da Solução Implementada}

A solução que desenvolvi traz várias vantagens significativas em relação ao sistema original. A principal delas é a conectividade - agora posso acessar os dados de controle de acesso de qualquer lugar do mundo, algo impensável com a controladora original que só funciona localmente. A escalabilidade também é um ponto forte, já que posso facilmente replicar essa solução para múltiplos pontos de acesso, todos enviando dados para o mesmo Firebase.

O design modular que criei facilita enormemente a manutenção e futuras atualizações. Se precisar trocar o método de armazenamento, por exemplo, posso modificar apenas o código do ESP8266 sem tocar no Arduino ou no hardware de leitura. O custo-benefício é excelente - gastei menos de R\$ 150,00 em componentes para adicionar funcionalidades que sistemas comerciais cobrariam milhares de reais.

\section{Discussão dos Resultados}

Os resultados obtidos confirmam a viabilidade técnica da solução proposta. A taxa de 100\% de sucesso na leitura dos cartões RFID demonstra a confiabilidade do módulo RDM6300 para esta aplicação. A capacidade de operar em paralelo com a controladora original, sem causar interferências, é particularmente importante para garantir a segurança e confiabilidade do sistema.

Um aspecto que merece destaque é a estabilidade demonstrada durante os testes de longa duração. Mesmo após uma semana de operação contínua, o sistema manteve seu desempenho sem degradação e sem aquecimento dos componentes, indicando uma implementação robusta e adequada para uso prolongado.

A verificação dos códigos impressos nos cartões foi fundamental para validar a precisão do sistema. Todos os valores lidos corresponderam exatamente aos códigos hexadecimais gravados, confirmando que o sistema interpreta corretamente o protocolo do RDM6300.

Os experimentos também revelaram a eficácia do sistema de reconexão automática implementado no ESP8266, que conseguiu se recuperar de quedas de energia sem perda de dados, demonstrando resiliência a falhas temporárias de infraestrutura.

%posicionamento dos elementos gráficos: figuras, gráficos e
%tabelas. Como estes elementos muitas vezes são grandes, aparece o
%dilema sobre o que fazer quando uma quebra de página deveria acontecer
%no meio do elemento. Há duas possibilidades:
%\begin{enumerate}
%\item O autor informa exatamente onde o elemento gráfico deve ficar no
%texto, evitando que quebras de páginas aconteçam no meio de um
%elemento. O problema com esta abordagem é que todo o trabalho de
%posicionamento pode ser perdido caso se inclua ou se exclua algum
%texto ou elemento.
%\item O editor de texto posiciona os elementos gráficos de forma a não
%deixar espaços em branco nas páginas. Estes elementos que podem ser
%posicionados pelo editor são conhecidos como \emph{elementos
%flutuantes}. O problema com esta abordagem é que o posicionamento
%adotado pode não corresponder às expectativas do autor.
%\end{enumerate}
%
%O \LaTeX\ oferece as duas possibilidades de posicionamento. Este
%capítulo apresenta exemplos de inclusão de elementos gráficos no
%texto, bem como algumas ferramentas externas ao \LaTeX\ que podem ser
%utilizadas para gerá-los.
%
%\section{Elementos flutuantes}
%\label{Sec:flutuantes}
%
%Para caracterizar uma parte do texto como sendo flutuante, ela deve ser
%delimitada por \verb|\begin{figure}| e \verb|\end{figure}| ou por
%\verb|\begin{table}| e \verb|\end{table}|. Apesar do que os nomes
%sugerem, nada obriga que o ambiente \texttt{figure} seja usado para
%delimitar figuras ou que o ambiente \texttt{table} seja usado para
%delimitar tabelas, embora esta seja a escolha quase sempre
%adotada. Estes dois ambientes são praticamente equivalentes, com as
%seguintes diferenças:
%\begin{itemize}
%\item os dois ambientes usam contadores diferentes para numerar os
%elementos flutuantes;
%\item os ambientes \texttt{figure} serão incluídos na
%\texttt{listoffigures}, enquanto os ambientes \texttt{table} serão
%incluídos na \texttt{listoftables};
%\item as legendas (\texttt{caption}'s) dos ambientes \texttt{figure}
%serão precedidas da palavra ``Figura \dots'', enquanto as legendas dos
%ambientes \texttt{table} serão precedidas da palavra ``Tabela \dots''.
%Estas duas palavras podem ser alteradas pelo autor.
%\end{itemize}
%Para ilustrar o fato de que estes ambientes podem conter virtualmente
%qualquer coisa, a figura~\ref{Fig:textoflutuante} contém um texto que
%foi tornado flutuante por ser incluído em um ambiente \texttt{figure}
%e as tabelas \ref{Tab:equacaoflutuante} e \ref{Tab:equacaoflutuante2}
%contêm expressões matemáticas flutuantes, incluídas em um ambiente
%\texttt{table}. A tabela (\texttt{table}) \ref{Tab:submultilinhas} na
%página \pageref{Tab:submultilinhas} também não contém uma tabela no
%sentido estrito do termo, mas sim uma linha de texto formada por duas
%\texttt{minipage}'s separadas por um espaço horizontal. A primeira
%\texttt{minipage} contém um trecho de código fonte e a segunda, o
%resultado produzido (uma expressão matemática multialinhada).
%
%\begin{figure}[tbp]
%\caption{Trecho de \emph{Os Lusíadas}, de Luis de Camões}
%\label{Fig:textoflutuante}
%% hrule - linha horizontal
%\hrule
%% As minipage's são muito úteis para se colocar duas coisas na mesma linha
%\begin{minipage}{0.45\linewidth}
%% flushleft - alinha à esquerda
%\begin{flushleft}
%As armas e os barões assinalados\\
%Que da ocidental praia lusitana\\
%Por mares nunca dantes navegados\\
%Passaram ainda além da Trapobana\\
%Em perigos e guerras esforçados\\
%Mais do que prometia a força humana\\
%Entre gente remota edificaram\\
%Novo reino, que tanto sublimaram
%\end{flushleft}
%\end{minipage}
%\hfill
%\begin{minipage}{0.45\linewidth}
%% flushright - alinha à direita
%\begin{flushright}
%E também as memórias gloriosas\\
%Daqueles reis que foram dilatando\\
%A Fé, o Império, as terras viciosas\\
%De África e Ásia andaram devastando,\\
%E aqueles que por obras valerosas\\
%Se vão da lei da morte libertando:\\
%Cantando espalharei por toda parte,\\
%Se a tanto me ajudar o engenho e arte.
%\end{flushright}
%\end{minipage}
%\hrule
%\end{figure}
%
%\begin{table}[bp]
%% As minipage's são muito úteis para se colocar duas coisas na mesma linha
%\begin{minipage}[b]{0.45\linewidth}
%\begin{center}
%\[
%ax^2 + bx + c = 0
%\]
%\end{center}
%\caption{Equação de segundo grau}
%\label{Tab:equacaoflutuante}
%\end{minipage}
%\hfill
%\begin{minipage}[b]{0.50\linewidth}
%\begin{center}
%\[
%x = \frac{-b\pm\sqrt{b^2-4ac}}{2a}
%\]
%\end{center}
%\caption{Raízes da equação da tabela~\ref{Tab:equacaoflutuante}}
%\label{Tab:equacaoflutuante2}
%\end{minipage}
%\end{table}
%
%É importante ressaltar que o que é numerado é o \texttt{caption} e não
%a \texttt{figure} ou a \texttt{table}. Portanto, o \texttt{label} deve
%ser colocado sempre após o \texttt{caption} ao qual ele se
%refere. Conforme ilustram as tabelas \ref{Tab:equacaoflutuante} e
%\ref{Tab:equacaoflutuante2}, uma mesma \texttt{figure} ou
%\texttt{table} pode ter mais de um ou nenhum \texttt{caption}.
%O \texttt{caption} pode ser colocado antes do conteúdo flutuante, como
%na figura \ref{Fig:textoflutuante}, ou depois, como nas tabelas
%\ref{Tab:equacaoflutuante} e \ref{Tab:equacaoflutuante2}. Nos
%documentos do PPgEE, o padrão é sempre posicionar o \texttt{caption}
%abaixo das figuras e das tabelas.
%
%\subsection{Posicionamento dos elementos flutuantes}
%\label{Sec:posicionamento}
%
%Em cada \verb|\begin{figure}| ou \verb|\begin{table}| pode-se incluir
%um parâmetro opcional com as opções de posicionamento para este
%elemento flutuante. Parâmetros adicionais de comandos \LaTeX\ são
%sempre fornecidos entre colchetes \texttt{[]}, enquanto os parâmetros
%obrigatórios aparecem entre chaves \verb|{}|. As opções disponíveis
%incluem as seguintes:
%\begin{itemize}
%\item[\tt h] O elemento pode ser posicionado na mesma posição em que ele
%aparece no código fonte do texto.
%\item[\tt t] O elemento pode ser posicionado no topo de uma página.
%\item[\tt b] O elemento pode ser posicionado no fim de uma página.
%\item[\tt p] O elemento pode ser incluído em uma página formada só por
%flutuantes.
%\item[\tt !] Normalmente o \LaTeX\ faz algumas considerações de ordem
%estética no posicionamento dos flutuantes, o que às vezes faz com que
%alguns elementos sejam posicionados muito longe de onde são citados,
%principalmente se você não incluir a opção \texttt{p}. Para fazer com
%que as considerações estéticas não sejam levadas em conta para um dado
%elemento, inclua a opção \texttt{!}.
%\end{itemize}
%
%\section{Tabelas em \LaTeX}
%\label{Sec:tabelas}
%
%Tabelas são construídas com comandos próprios do \LaTeX, notadamente o
%ambiente \texttt{tabular}. Nada obriga a que o ambiente
%\texttt{tabular} esteja sempre posicionado em um elemento
%flutuante. Se você quiser impor que uma tabela fique obrigatoriamente
%em uma determinada posição do texto, basta não colocar o
%\texttt{tabular} dentro de um \texttt{table}. Tabelas podem até ser
%incluídas no meio de uma frase.  Por exemplo, eu posso dizer que se um
%jogo da velha está na configuração \textsf{\tiny\begin{tabular}{c|c|c}
%x & & x \\ \hline & & o \\ \hline x & o & \end{tabular}} e se o
%jogador ``\textsf{x}'' sabe jogar, então o jogador ``\textsf{o}'' irá
%perder, independentemente da jogada que faça.
%
%O ambiente \texttt{tabular} tem um parâmetro obrigatório que indica o
%número de colunas da tabela e o posicionamento dos objetos em cada
%coluna. Por exemplo, uma tabela criada com \verb|\begin{tabular}{lcr}|
%terá três colunas; o texto será alinhado à esquerda na primeira
%coluna, centralizado na segunda e alinhado à direita na
%terceira. Podem ser incluídos objetos que ocupam mais de uma linha
%(comando \texttt{multirow}) ou mais de uma coluna (comando
%\texttt{multicolumn}). Neste último caso, também é possível mudar o
%alinhamento do texto. Exemplos podem ser vistos nas tabelas
%\ref{Tab:multilinhas} e \ref{Tab:submultilinhas}, na
%página~\pageref{Tab:multilinhas}.
%
%Com o pacote \texttt{tabularx}, além das opções normais de
%posicionamento de colunas (\texttt{lcr}), pode-se incluir
%automaticamente um texto qualquer antes de cada elemento da coluna
%(\verb|>{}|). Este recurso foi utilizado nas tabelas
%\ref{Tab:multilinhas} e \ref{Tab:submultilinhas} para fazer
%com que todos os textos de algumas colunas fossem automaticamente
%escritos na fonte \texttt{tt}. Além disso, podem-se criar colunas de
%largura fixa e/ou de largura que se ajustam para que a tabela ocupe
%toda a largura desejada, além do estilo tradicional de coluna que
%assume a largura suficiente para conter seus elementos. Exemplos de
%colunas com diferentes larguras e alinhamentos podem ser vistos na
%tabela \ref{Tab:larguracolunas}.
%
%\begin{table}[htbp]
%\begin{tabularx}{\linewidth}{|p{3cm}|X|l|} \hline
%COLUNA p & COLUNA X & COLUNA l \\ \hline
%Largura fixa (não depende do conteúdo) &
%Expandível &
%Ajustável \\ \hline
%Alinhada no topo &
%Alinhada à esquerda &
%Alinhada à esquerda \\ \hline
%\end{tabularx}
%\\[0.5cm]
%\begin{tabularx}{\linewidth}{|b{3cm}|C|r|} \hline
%COLUNA b & COLUNA C (ver \texttt{comandos.tex}) & COLUNA r \\ \hline
%Largura fixa (não depende do conteúdo) &
%Expandível &
%Ajustável \\ \hline
%Alinhada na base &
%Centralizada &
%Alinhada à direita \\ \hline
%\end{tabularx}
%\caption{Tabelas com colunas de diferentes larguras e alinhamentos}
%\label{Tab:larguracolunas}
%\end{table}
%
%\section{Figuras em \LaTeX}
%\label{Sec:figuras}
%
%As figuras (imagens, desenhos, gráficos, etc.) devem ser produzidas
%por ferramentas externas ao \LaTeX, salvas em um arquivo e inseridas
%no texto usando o comando \texttt{includegraphics}. Da mesma forma
%que as tabelas, as figuras podem ser flutuantes, caso sejam
%inseridas dentro de um ambiente \texttt{figure}, ou ter uma posição
%fixa no texto (como aqui: \includegraphics{textuais/04-figuras/figuras/eu}).
%
%O formato em que você deve salvar os arquivos das figuras para que
%possa incluí-las no texto depende de como você pretende compilar
%o código fonte:
%\begin{itemize}
%\item se o texto vai ser compilado com \texttt{latex}, todos os
%arquivos devem estar no formato EPS (\emph{Encapsulated PostScipt});
%\item se o texto vai ser compilado com \texttt{pdflatex}, os
%arquivos devem estar nos formatos PDF ou JPEG (outros formatos são
%aceitos, mas estes são os recomendáveis).
%\end{itemize}
%É aconselhável que você não inclua a terminação no nome do arquivo que
%é parâmetro para o comando \texttt{includegraphics}. Isto porque, de
%acordo com a forma como o texto está sendo compilado, o \LaTeX\
%acrescenta a terminação adequada. Por exemplo, caso seu texto inclua o
%comando \verb|\includegraphics{eu}|, o \LaTeX\ procurará o arquivo
%\texttt{eu.eps} caso esteja sendo chamado via \texttt{latex} ou um dos
%arquivos \texttt{eu.pdf} ou \texttt{eu.jpg} caso esteja sendo chamado
%via \texttt{pdflatex}.
%
%As figuras podem ser divididas em dois grandes grupos:
%\begin{itemize}
%\item As imagens e fotos, que normalmente correspondem a visões reais
%do mundo e são obtidas por câmeras digitais ou
%assemelhados. Caracterizam-se por conterem grandes quantidades de
%nuances, texturas e cores.
%\item As figuras sintéticas, normalmente produzidas utilizando
%\emph{softwares} dedicados. Geralmente contêm figuras geométricas
%(linhas, quadrados, etc.), textos e poucas cores e texturas. Neste
%grupo, para efeito de discussão das ferramentas de produção, podem-se
%identificar duas categorias:
%\begin{itemize}
%\item Os desenhos e esquemas: diagramas de blocos, organogramas e
%fluxogramas, representações esquemáticas, etc.
%\item Os gráficos: representações gráficas de valores ou funções
%matemáticas.
%\end{itemize}
%\end{itemize}
%
%\subsection{Imagens e fotos}
%\label{Sec:imagens}
%
%As imagens e fotos normalmente só podem ser armazenadas em formatos
%que representam cada \emph{pixel} da imagem separadamente,
%eventualmente com algum tipo de compressão. Os formatos JPEG, GIF,
%TIF, PNM (PBM, PGM ou PPM), BMP (Bitmap) e PNG, entre outros, são
%todos desta categoria.  Se sua figura está em algum destes formatos,
%você deve convertê-la para EPS (se usar \texttt{latex}) ou para JPEG
%(se usar \texttt{pdflatex}) para poder incluí-la no documento \LaTeX.
%
%A quase totalidade dos \emph{softwares} de visualização de imagens
%permite salvá-las em múltiplos formatos, geralmente incluindo JPEG e
%EPS. No Unix, você dispõe ainda de vários programas para fazer a
%conversão em comandos de linha: \texttt{jpegtopnm},
%\texttt{pnmtojpeg}, \texttt{pnmtops}, \texttt{gif2ps},
%\texttt{giftopnm}, \texttt{tiff2ps}, \texttt{tifftopnm},
%\texttt{bmptopnm} e \texttt{pngtopnm}, entre outros.
%
%A figura \ref{Fig:belmonte} mostra um exemplo de inclusão de uma
%imagem no texto \LaTeX.
%
%\begin{figure}[htbp!] \begin{center}
%% fbox faz uma borda ao redor do seu argumento
%\fbox{\includegraphics[width=0.75\linewidth]{textuais/04-figuras/figuras/belmonte}}
%\caption{Exemplo de imagem real}
%\label{Fig:belmonte}
%\end{center} \end{figure}
%
%\subsection{Figuras sintéticas}
%\label{Sec:figsinteticas}
%
%As figuras sintéticas podem ser armazenadas em formato
%\emph{pixel}-a-\emph{pixel}, como se fossem uma imagem, ou em
%formato vetorial. No formato vetorial as primitivas que formam a
%figura (linhas, textos, etc.) são descritas pelos parâmetros que as
%caracterizam (ponto de início e fim, \emph{string} e posição do texto,
%etc.). As figuras em formato vetorial são mais adequadas pois
%usualmente correspondem a arquivos menores e a qualidade da imagem
%não sofre perdas ao se aumentar ou diminuir o tamanho da figura.
%
%Para inclusão no \LaTeX, os formatos PDF e EPS são os únicos que podem
%representar figuras no formato vetorial. Nem toda figura salva nestes
%formatos, entretanto, é necessariamente vetorial, pois tanto o PDF
%quanto o EPS podem representar tanto figuras em formato
%\emph{pixel}-a-\emph{pixel} quanto figuras em formato vetorial. Para
%que sua figura seja vetorial, é necessário que o \emph{software} que a
%gerou tenha a capacidade de produzi-las.
%
%Para demonstrar a melhor qualidade das figuras em formato vetorial,
%nas figuras \ref{Fig:bigvetorial} e \ref{Fig:bigbitmap} se mostra em
%tamanho natural um mesmo diagrama nos formatos vetorial e de
%\emph{pixels}. Nas figuras \ref{Fig:bigvetorialreduzida} e
%\ref{Fig:bigbitmapreduzida} estas mesmas figuras são apresentadas
%com uma redução de 50\%, utilizando o parâmetro \texttt{scale} do
%\texttt{includegraphics}. Já nas figuras \ref{Fig:smallvetorial} e
%\ref{Fig:smallbitmap} o diagrama original foi reduzido, de forma que
%seu tamanho natural é menor. Nas figuras
%\ref{Fig:smallvetorialampliada} e \ref{Fig:smallbitmapampliada}
%este diagrama pequeno está aumentado de um fator arbitrário, calculado
%pelo \texttt{includegraphics} para que a imagem ocupe toda a largura
%da linha.
%
%\begin{figure}[htbp!] \begin{center}
%\includegraphics{textuais/04-figuras/figuras/bigvetorial}
%\caption{Figura vetorial grande em tamanho natural}
%\vspace{6mm}
%\label{Fig:bigvetorial}
%\includegraphics{textuais/04-figuras/figuras/bigbitmap}
%\caption{Figura \emph{pixel}-a-\emph{pixel} grande em tamanho natural}
%\label{Fig:bigbitmap}
%\vspace{6mm}
%\includegraphics[scale=0.5]{textuais/04-figuras/figuras/bigvetorial}
%\caption{Figura vetorial grande em tamanho reduzido}
%\label{Fig:bigvetorialreduzida}
%\vspace{6mm}
%\includegraphics[scale=0.5]{textuais/04-figuras/figuras/bigbitmap}
%\caption{Figura \emph{pixel}-a-\emph{pixel} grande em tamanho reduzido}
%\label{Fig:bigbitmapreduzida}
%\end{center} \end{figure}
%
%\begin{figure}[htbp!] \begin{center}
%\includegraphics{textuais/04-figuras/figuras/smallvetorial}
%\caption{Figura vetorial pequena em tamanho natural}
%\label{Fig:smallvetorial}
%\vspace{6mm}
%\includegraphics{textuais/04-figuras/figuras/smallbitmap}
%\caption{Figura \emph{pixel}-a-\emph{pixel} pequena em tamanho natural}
%\label{Fig:smallbitmap}
%\vspace{6mm}
%\includegraphics[width=\linewidth]{textuais/04-figuras/figuras/smallvetorial}
%\caption{Figura vetorial pequena em tamanho ampliado}
%\label{Fig:smallvetorialampliada}
%\vspace{6mm}
%\includegraphics[width=\linewidth]{textuais/04-figuras/figuras/smallbitmap}
%\caption{Figura \emph{pixel}-a-\emph{pixel} pequena em tamanho ampliado}
%\label{Fig:smallbitmapampliada}
%\end{center} \end{figure}
%
%Nota-se que no formato vetorial as
%linhas mantêm a espessura mesmo quando se fazem
%ampliações ou reduções. Já no formato de \emph{pixels}
%as linhas ficam mais claras (cinzas, ao invés de pretas) após as
%reduções e mais grossas após as ampliações, além de uma perda geral
%de definição da imagem.
%
%\section{Ferramentas para desenhos e esquemas}
%\label{Sec:desenhos}
%
%Existem diversas ferramentas para fazer desenhos, mas muitas delas
%apenas salvam a figura gerada em formatos \emph{pixel}-a-\emph{pixel}.
%No Unix, pode-se utilizar o \texttt{xfig}, que exporta imagens em
%muitos formatos, inclusive nos vetoriais (PDF e EPS). Os diagramas das
%figuras \ref{Fig:bigvetorial} a \ref{Fig:smallbitmapampliada} foram
%desenhados e exportados no \texttt{xfig}. O arquivo fonte
%correspondente é o \texttt{diagrama.fig}, no diretório
%\texttt{figuras}.
%
%A possibilidade de salvar figuras em modo vetorial impõe que alguns
%recursos para desenho de imagens não sejam oferecidos. Um deles é o
%desenho a mão-livre, já que seria impossível descrever a curva obtida
%em termos de figuras geométricas básicas. Outro recurso inexistente é
%o de preencher uma região com uma determinada cor. Esta última
%limitação muitas vezes pode ser contornada utilizando-se a noção de
%profundidade.  Por exemplo, para desenhar uma figura vazado e
%preenchido de azul, pode-se desenhar a figura externa preenchido de
%azul sobre o qual se desenha a figura interna preenchido de branco,
%como mostram os exemplos da figura~\ref{Fig:circulo}.
%
%\begin{figure}[htb] \begin{center}
%\includegraphics{textuais/04-figuras/figuras/circulo}
%\caption{Preenchimento de figuras utilizando diferentes profundidades}
%\label{Fig:circulo}
%\end{center} \end{figure}
%
%A noção de profundidade no \texttt{xfig} foi exaustivamente utilizada
%para desenhar os símbolos da UFRN e do PPgEE que podem ser vistos na
%página de rosto deste documento. Os arquivos \texttt{xfig}
%correspondentes são \texttt{UFRN.fig} e \texttt{PPgEE.fig}. Ela também
%pode ser utilizada para mesclar imagens com figuras sintéticas, como
%na figura \ref{Fig:pensador} (veja arquivo \texttt{figuras/pensador.fig}).
%
%\begin{figure}[htb] \begin{center}
%\includegraphics{textuais/04-figuras/figuras/pensador}
%\caption{Imagem mesclada com elementos sintéticos}
%\label{Fig:pensador}
%\end{center} \end{figure}
%
%Outra possibilidade oferecida pelo \texttt{xfig} é a inclusão de comandos
%\LaTeX\ dentro da figura. Para utilizar este recurso,
%marque no \texttt{xfig} os textos que devem ser interpretados como
%comandos \LaTeX\ com o \emph{flag} \texttt{special} e exporte a figura
%no modo \emph{Combinado PS/Latex} ou \emph{Combinado PDF/Latex}. Veja
%um exemplo na figura \ref{Fig:combinado}; note que o arquivo é incluído com
%\verb|\input{}| e não com \verb|\includegraphics{}|.
%
%% Note que foi redefinido um comando aqui no texto para ser incluído
%% na figura. Isto é para evitar digitação de expressões LaTeX muito
%% grandes dentro do xfig
%\newcommand{\formulagrande}{$\frac{G_3G_4}{1-G_3G_4H_1}$}
%\begin{figure}[htb] \begin{center}
%%\input{figuras/combinado.pstex_t} % Se usar latex
%\input{textuais/04-figuras/figuras/combinado.pdftex_t} % Se usar pdflatex
%\caption{Figura incluindo comandos \LaTeX}
%\label{Fig:combinado}
%\end{center} \end{figure}
%
%\section{Ferramentas para gráficos}
%\label{Sec:graficos}
%
%Gráficos devem ser gerados com aplicativos capazes de exportar o
%resultado nos formatos EPS ou PDF, preferencialmente em formato
%vetorial. Os conhecidos programas \emph{Scilab} e \emph{Matlab} têm
%esta capacidade. Se você deseja algo mais simples, a ferramenta
%\textit{GNUplot} é uma das mais utilizadas no Unix para a geração de
%gráficos de funções matemáticas.
%
%Uma vez gerados, gráficos são inseridos no texto tal como figuras. A
%figura~\ref{fig:grafico} apresenta um gráfico gerado através do
%comando de linha \texttt{gnuplot grafico.gnuplot}. Este arquivo
%\texttt{grafico.gnuplot}, que contém uma série de comandos do
%\textit{GNUplot}, está no diretório \texttt{figuras}.
%
%\begin{figure}[htbp]
%\centering
%\includegraphics{textuais/04-figuras/figuras/grafico}
%\caption{Exemplo de gráfico de funções matemáticas}
%\label{fig:grafico}
%\end{figure}
%
%\section{Conclusões}
%
%Ferramentas de desenho capazes de gerar a saída em formato vetorial
%são mais difíceis de usar e parecem ser dotadas de menos recursos do
%que outras que só exportam seus resultados como imagens de
%\emph{pixels}.  Isto se deve à necessidade de descrever todos os
%elementos da imagem sob a forma de primitivas parametrizáveis para
%permitir que elas sejam escaláveis à vontade e exportáveis para
%qualquer formato desejado.
%
%Entretanto, a qualidade visual das figuras obtidas e a sua
%reusabilidade é muito maior. A comparação é aproximadamente a mesma
%que a entre textos produzidos em \LaTeX\ e em editores gráficos. Desta
%forma, na medida do possível, tente conjugar a escrita do documento
%\LaTeX\ com a utilização de alguma ferramenta de desenho vetorial.
%
%% LocalWords:  editadas PS
